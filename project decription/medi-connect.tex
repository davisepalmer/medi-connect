\documentclass{article}

% Packages for styling
\usepackage[utf8]{inputenc}
\usepackage{geometry}
\usepackage{titlesec}
\usepackage{graphicx}
\usepackage{lipsum} % For dummy text, you can remove this in your actual document

% Customizing margins
\geometry{a4paper, margin=1in}

% Customizing section headings
\titleformat{\section}{\normalfont\Large\bfseries}{\thesection}{.75em}{}
\titlespacing*{\section}{0pt}{\baselineskip}{\baselineskip}

\title{Medi-Connect}
\author{Leela Ravoori, Shasshank Sethuraman, Sahran Ashoor, and Davis Palmer}
\date{\today}

\begin{document}

\maketitle
\thispagestyle{empty} % Removes page number from this page

\begin{figure}[h]
    \centering
    \includegraphics[width=0.15\textwidth]{logoFinal.png}
\end{figure}

\section*{Project Root}

Our Hospital Resource Management System streamlines resource sharing among hospitals, optimizing bed allocation, medicine distribution, and facilitating second opinions from specialists. Hospitals can easily view and offer available beds, share surplus medicines, and seek expert advice through secure channels. Additionally, the system enables coordinated emergency responses and provides valuable insights through data analytics. By promoting collaboration and efficiency, our software enhances patient care, reduces costs, and strengthens disaster preparedness within the healthcare system.

\section*{Development Process}

Our development process started with throwing around ideas and brainstorming. We came in with the idea that we wanted to target the healthcare track and eventually landed on the idea of the amount of resources lost in hospitals due to expiration dates, and how many hospitals have resources they \emph{can} share. Our software enables hospitals to share their extra resources allowing for impoverished hospitals to have the assistance they need while making it easier for private, highly-corporate, and really any institution to share their resources.

\section*{Challenges and Obstacles}

As a team of students who have never met prior to the event, all of which hold very little experience performing
in hackathon settings, if at all, were tasked with quickly adapting to an unfamiliar programming environment.
With a thirty-six hour clock soon to start ticking, our team chemistry had proven itself as we worked together
to familiarize ourselves with the Github environment. Through the development of our codebase, we faced frequent
roadblocks in honing common practices and etiquette when managing source requests, as unsychronized/unnecessary
push and pull requests easily shattered our mental timelines, abrubtly halting our flow of productivity. It took
time to familiarize ourselves with these environments, but as our communication developed, we established
a streamlined train of collaboration.

\section*{Acknowledgements}
From all of us, we thank you for inviting us to Georgia Tech's Hacklytics, and it was great to be a part of this event. We hope you like our project and idea, we have a lot of ideas for it planned for the future!

\end{document}
